%%%%%%%%%%%%%%%%%%%%%%%%%%%%%%%%%%%%%%%%%%%%%%%%%%%%%%%%%%%%%%%%%
%%%%%%%%%%%%%%%%%%%%%%%%%%%%%%%%%%%%%%%%%%%%%%%%%%%%%%%%%%%%%%%%%
%%%%%%%%%%%%%%%%%%%%%%%%%%%%%%%%%%%%%%%%%%%%%%%%%%%%%%%%%%%%%%%%%

% STM 2007-04-13  split from previous version

\chapter{Data Examination and Editing}
\label{chapter:edtool}

%%%%%%%%%%%%%%%%%%%%%%%%%%%%%%%%%%%%%%%%%%%%%%%%%%%%%%%%%%%%%%%%%
\section{Plot the Data}
\label{section:edtool.pl}

CASA uses the {\tt matplotlib} plotting library to display its plots.
You can find information on {\tt matplotlib} at
\url{http://matplotlib.sourceforge.net/}.

\subsection{Plot Symbols}
\label{section:edtool.pl.symb}

The {\tt plotsymbol} defines both the line or
symbol for the data being drawn as well as the color; from the
matplotlib online documentation (e.g., type {\tt pl.plot?} for help):

\small
\begin{verbatim}
    The following line styles are supported:
        -     : solid line
        --    : dashed line
        -.    : dash-dot line
        :     : dotted line
        .     : points
        ,     : pixels
        o     : circle symbols
        ^     : triangle up symbols
        v     : triangle down symbols
        <     : triangle left symbols
        >     : triangle right symbols
        s     : square symbols
        +     : plus symbols
        x     : cross symbols
        D     : diamond symbols
        d     : thin diamond symbols
        1     : tripod down symbols
        2     : tripod up symbols
        3     : tripod left symbols
        4     : tripod right symbols
        h     : hexagon symbols
        H     : rotated hexagon symbols
        p     : pentagon symbols
        |     : vertical line symbols
        _     : horizontal line symbols
        steps : use gnuplot style 'steps' # kwarg only
    The following color strings are supported
        b  : blue
        g  : green
        r  : red
        c  : cyan
        m  : magenta
        y  : yellow
        k  : black
        w  : white
    Line styles and colors are combined in a single format string, as in
    'bo' for blue circles.
\end{verbatim}
\normalsize

\subsection{Pylab - {\tt matplotlib} library interface}
\label{section:edtool.pl.pylab}

For even more functionality, you can access the {\tt pl} tool directly
using {\tt pylab} functions that allow one to annotate, alter, or add
to any plot displayed in the matplotlib plotter (e.g. {\tt
plotxy}). {\tt matplotlib} commands can be found at:

\url{http://matplotlib.sourceforge.net/pylab\_commands.html}

The {\tt pl.clf()} method is perhaps the most useful native command as it will
clear the contents of the plotter frame.

A quick synopsis of relevant commands for altering plots are found
below. 

\small
\begin{verbatim}
  pl.axhline - draw a horizontal line across axes
  pl.axvline - draw a vertical line across axes
  pl.axhspan - draw a horizontal bar across axes
  pl.axvspan - draw a vertical bar across axes
  pl.clf     - clear the current figure
  pl.close   - close the plotter (subsequent plots will reopen a plotter window)
  pl.ion     - turn interaction mode on (default - this indicates that any plotting command
               will be seen immediately after the command)
  pl.ioff    - turn off interaction mode (a 'show' command is required to see commands after
               this mode has been enabled)
  pl.savefig - save the current figure
  pl.subplot - make a subplot (numrows, numcols, axesnum)
  pl.text    - add some text at location (x,y)
  pl.title   - add/change the title
  pl.xlim    - set/get the xlimits
  pl.ylim    - set/get the ylimits
  pl.xlabel  - add/change the xlabel
  pl.ylabel  - add/change the ylabel
\end{verbatim}
\normalsize

In addition, there are a range of mathematical functions provided
(trigonometric, matrix algebra, logs, etc). Again, see the pylab
documentation for help.

%%%%%%%%%%%%%%%%%%%%%%%%%%%%%%%%%%%%%%%%%%%%%%%%%%%%%%%%%%%%%%%%%
\section{Data Flagging}
\label{section:flagging}

%%%%%%%%%%%%%%%%%%%%%%%%%%%%%%%%%%%%%%%%%%%%%%%%%%%%%%%%%%%%%%%%%
%%%%%%%%%%%%%%%%%%%%%%%%%%%%%%%%%%%%%%%%%%%%%%%%%%%%%%%%%%%%%%%%%
