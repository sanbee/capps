\documentclass[11pt]{article}
%\documentstyle[12pt,amsmath]{article}
%\usepackage{html}
\usepackage{epsf}
\usepackage{amsmath}
\usepackage[dvips]{graphicx, color}  % The figure package
%\usepackage{utopia} 
\def\vec#1{\ensuremath{\mathchoice{\mbox{\boldmath$\displaystyle#1$}}
{\mbox{\boldmath$\textstyle#1$}}
{\mbox{\boldmath$\scriptstyle#1$}}
{\mbox{\boldmath$\scriptscriptstyle#1$}}}}
%
\def\tens#1{\ensuremath{\mathsf{#1}}}
\newcommand{\arcmin}      {{^\prime}}
\newcommand{\arcsec}      {{^{\prime\prime}}}
\newcommand{\arcsecSq}    {{\mathrm{arcsec^2}}}
\newcommand{\SVJ}         {{\tt SolvableVisJones}}
\newcommand{\VJ}          {{\tt VisJones}}
\newcommand{\TVVJ}        {{\tt TimeVarVisJones}}
\newcommand{\VE}          {{\tt VisEquation}}
\newcommand{\FTM}         {{\tt FTMachine}}
\def\MSVJ#1{\SVJ{\tt::#1}}
\def\MVE#1{\VE{\tt::#1}}
\def\MFTM#1{\FTM{\tt::#1}}
%
\setlength{\textheight}{23.0cm}
\setlength{\textwidth}{16.00cm}
\setlength{\topmargin}{-0.1cm}
\setlength{\oddsidemargin}{-0.0cm}
% \setlength{\evensidemargin}{1.5cm}
%\setlength{\parskip}{5pt}
%\setlength{\parindent}{20pt}

%\evensidemargin -0.7cm
%\oddsidemargin 1.5cm
%\textwidth 13cm
%\topmargin -1.5cm
%\textheight 23cm
\parskip 1ex    % White space between paragraphs amount
\pagestyle{myheadings}
\markboth{Noise propagation}{S. Bhatnagar}

\usepackage{mathptmx}
\begin{document}
\title{Error propagation in solvers for antenna based parameters}
\author{S.Bhatnagar$^{1}$, D. Oberoi$^{2}$\\ $^{1}$NRAO, Socorro, $^{2}$MIT Haystack Observatory}
\date{Nov. 25, 2009}
\maketitle
\normalsize


% \begin{abstract}
% \end{abstract}

\section{Introduction}

In this note we analyze the validity of the conventional wisdom that
{\it "if we can see an error in the image domain, we can correct for
  it"}.  In practice, we are forced to use algorithms to solve and
correct for antenna based errors - particularly when the errors we see
might be coming from inherently antenna based effects. Use of baseline
based corrections for such errors will not lead to scientifically
objective results (images).  % {\bf I wish to borrow your mind to prove
%   or disprove} the following basic deduction (which I am attempting to
% make more precise in the following section):
The basic premise that we will make more precise in the following
sections is:
\begin{description}
\item Noise on the solution for antenna based parameters is
  proportional to ${N^{-\frac{1}{2}}_{ant}}$ while the noise in the
  image domain is proportional to $N^{-1}_{ant}$.
\item Therefore, in the space of telescope parameters, there must be a
  region where it might be fundamentally impossible to achieve imaging
  dynamic range, as determined {\it only} by the telescope System
  Equivalent Flux Density (SEFD).
\end{description}

If the telescope parameters (embodied in SEFD) are such that the
signal-to-noise ratio available to solve for the antenna based
parameters is not sufficient over the solution interval and bandwidth,
then {\it fundamentally} we cannot achieve thermal noise limit in the
image domain.  % If there is no catch in this, this can shed some light
% on what we should expect from our current telescopes.  
This issue is of acute importance for the SKA where scientific goals
are pushing the imaging performance to one limit while cost
considerations are pushing the antenna element design to another
(opposite) limit.  If the following analysis is correct, then
understanding the coupling between these axis (noise in the image and
antenna element cost) more precisely is important to determine the
range of do-able science.

\section{Noise propagation}

Assuming identical antennas, only additive random noise and {\it only}
antenna based corruptions, the observed data from
baseline $i-j$ can be modeled as
\begin{equation}
V^{Obs}_{ij} = f(\left\{a_i\right\}, I^M) + \sigma_{ij}
\end{equation}
where $\left\{a_i\right\}$ is the set of antenna based parameters, $I^M$ is the
model of the sky emission and $f$ is a function that models the
observed data.  $\sigma_{ij}$ is the additive random noise
and is related to the telescope and observation parameters as
\begin{eqnarray}
\sigma_{ij} &=& \frac{2 k_b T_{sys}}{\eta_a A \sqrt{\nu_{corr} \tau_{corr}}}\\
&=& \frac{SEFD}{\sqrt{\nu_{corr} \tau_{corr}}}
\end{eqnarray}
where $\nu_{corr}$ and $\tau_{corr}$ are the correlator
bandwidth and integration time (strictly speaking, the above equation
also assumes the noise on the data samples separated by
$\nu_{corr}$ in frequency and $\tau_{corr}$ in time is
statistically independent).

Since $\sigma_{ij}$ is gaussian random noise, all algorithms we use to
solve for the parameters of $f$ ($\{\vec{a}\}$ and $I^M$) minimize the
$\chi^2$ function with respect to the parameters, using iterative
Least Square (LSQ) minimization algorithms (SelfCal and most image
deconvolution algorithms are Steepest Descent minimization algorithms)

In an iterative LSQ minimization algorithm, the diagonal of the
Covariance matrix is given by
\begin{equation}
 C_{kk} = \left[ \sum_{ij}\frac{1}{\sigma_{ij}^2} 
   \left(
     \frac{\partial f(\left\{ a \right\})}{\partial a_k} \right)^2\right]^{-1}
\end{equation}
where the summation goes over all baselines (denoted by $ij$).  For a
solution interval of $\tau_{sol}$ and bandwidth of $\nu_{sol}$ over
which the parameters can be assumed to be stable, integration over
time and frequency is done to increase the signal-to-noise ratio per
parameter.  The resulting $C_{kk}$ is given by
\begin{equation}
\label{CKKAvg}
 C_{kk} = \left[ \sum_{ij}\sum_{\tau_{sol},\nu_{sol}}\frac{1}{\sigma_{ij}^2} 
   \left(
     \frac{\partial f(\left\{ a \right\})}{\partial a_k} \right)^2\right]^{-1}
\end{equation}
Assuming the parameter space ($\left\{ a \right\}$) is orthogonal, the
noise on the solved parameter (here $a_k$) is related to the diagonal
of the covariance matrix as $\sigma(a_k) = \sqrt{C_{kk}}$.

When $f(\left\{a_i\right\}, I^M)=S g_i g_j^*$ where $S$ is the total
flux in the beam and $g_i$ is the antenna based complex multiplicative
gain, averaging over all baselines, denoted by $ij$ in Eq.~\ref{CKKAvg},
reduces to averaging over all antennas since
\begin{equation}
  \frac{\partial (g_ig_j^*)}{\partial g_k} = 
  \begin{cases}
    g_j^*, &\text{for $k=i$}\\
    0,    &\text{otherwise}
  \end{cases}
\end{equation}
Equation~\ref{CKKAvg} becomes
\begin{equation}
C(g_i) = \left[ \sum^{N_a}_{j} \sum_{\tau_{sol},\nu_{sol}}\left(\frac{Sg^*_j}{\sigma_{ij}}\right)^2  \right]^{-1}
\end{equation}

For an average antenna based gain of $\overline{g}$ and a solution
bandwidth and time interval of $\nu_{sol}$ and $\tau_{sol}$
respectively,

\begin{equation}
\label{DISigma}
\sigma(g_i) = \frac{\sigma_{ij}}{S \overline{g}}\frac{1}{\sqrt{N_a \nu_{sol} \tau_{sol}}}
\end{equation}

Multiplicative antenna based complex gains are direction-independent
and the noise (or the relative error) in solving for such terms is
clearly inversely proportional to the total flux $S$. In general, the
antenna based terms being solved for are {\it directionally dependent}
(DD).  Antenna primary beam pattern (PB) for a measurement from
baseline $i-j$ is given by $\sqrt{V_i V_j^*}$ where $V_i$ is the
antenna far-field voltage pattern.  Taking the example of antenna PBs
parametrized with the antenna pointing errors $l_i$, noise in the
solutions is given by
\begin{eqnarray}
\label{DDSigma1}
\sigma(l_i) &=& \frac{\sigma_{ij}}{\int \frac{\partial V_{P_i}}{\partial l_i} V^*_{P_j} I^M(\vec{l}) e^{2
    \pi \iota \vec{u}.\vec{l}} d\vec{l}}~~\frac{1}{\sqrt{N_a
    \nu_{sol}\tau_{sol}}}\\
\label{DDSigma2}
&=& \frac{2 k_b}{E_l}\frac{T_{sys}}{\eta_a A\sqrt{\nu_{corr}\tau_{corr}}
  \sqrt{N_a \nu_{sol} \tau_{sol}}}\\\nonumber
\\ \nonumber
\\
where~~~E_l &=& \int \frac{\partial V_{P_i}}{\partial l_i} V^*_{P_j} I^M(\vec{l}) e^{2
    \pi \iota \vec{u}\cdot \vec{l}} d\vec{l}
\end{eqnarray}

and $V_{P_i}$ is the antenna far-field voltage pattern.  For direction
independent terms, the integral in the numerator reduces to the total
flux in the beam, as in Eq.~\ref{DISigma}.

\section{Conclusions}
The first term in Eq.~\ref{DDSigma2} depends purely on the sky
emission, while the second term depends on the telescope hardware
design parameters ($N_a, \eta_a, A$ and $T_{sys}$) as well as on the
telescope software design parameters ($\nu_{sol}$ and $\tau_{sol}$).
Assuming azimuthally symmetric PBs, or equivalently, either
non-rotating PBs or apriori knowledge of the PB and being able correct
for PB rotation in the imaging software, these expressions lead to the
following inferences:
\begin{enumerate}
\item Taking the example of antenna pointing errors, for a pointing
  error of $\delta l^\prime_i$ for antenna $i$, the signal-to-noise ratio
  available to solve for the pointing error is given by
\begin{eqnarray}
  SNR_{l_i} &=& \frac{\nabla E_l}{2 k_b}~~ \frac{\eta_a
    A \sqrt{N_a \nu_{sol} \tau_{sol}}}{T_{sys}}\sqrt{\nu_{corr}\tau_{corr}}\\ 
  &=& \frac{\nabla E_l \sqrt{N_a\nu_{sol}\tau_{sol}}}{SEFD}\sqrt{\nu_{corr}\tau_{corr}}\\\nonumber
  \\ \nonumber
  \\ 
  \label{DELTA-E}
  where~~~\nabla E_l &=& \int \left(\delta l^\prime_i \frac{\partial
      V_{P_i}}{\partial l_i}\right) V^*_{P_j} I^M(\vec{l}) e^{2\pi
    \iota \vec{u}\cdot \vec{l}} d\vec{l}
\end{eqnarray}
Therefore, a telescope for which $\nabla E_l < SEFD/\left(\sqrt{N_a\nu_{sol}\tau_{sol}}\sqrt{\nu_{corr}\tau_{corr}}\right)$ for
{\it any} antenna based parameter, can {\it never} achieve the
"thermal noise" limit as defined {\it only} by the SEFD parameter in
the image domain.

\item The maximum error-signal ($\nabla E_l$) comes from the half
  power point of the PB, where typically the derivative is highest.
  Assuming that a gaussian describes the mainlobe of the PB
  accurately, the error signal $\nabla E_l \sim 55\times 10^{-5} \delta l_{\%}
  S_{Jy} ~Jy$ where $\delta l_\%$ is the antenna pointing error as
  a percentage of the antenna HPBW and $S_{Jy}$ is the total flux in
  Jy at the half power point of the PB.  The maximum pointing error
  that one can expect to be able to solve for is given by 
\begin{equation}
  \delta l^{max}_\% = \frac{SEFD}{S_{Jy} 55\times 10^{-5}\sqrt{N_a \nu_{sol} \tau_{sol}}\sqrt{\nu_{corr}\tau_{corr}}}
\end{equation}
For the EVLA, $T_{sys} = 70K,~N_a=27,~A=491~m^2$ and assuming that the
pointing errors are constant in time and frequency so that
$\nu_{sol}=1~GHz$ and $\tau_{sol} = 8^h$, for $S_{Jy}$ in the range
1~Jy -- 10~mJy, $\delta l^{max}_\% \sim 0.02 - 2.5\%$ .  The maximum
solvable antenna pointing error will be proportionately lower for
telescopes with smaller number of antennas, or smaller collecting area
or higher $T_{sys}$.  For the EVLA, the range of $\delta l^{max}_\%$
will be somewhere between these limits (i.e., at L-Band, depending on
the flux in the beam, it should be possible to solve for pointing
errors between 0.36 -- 36 arcsec.  More precise estimate will require
evaluation of the intergral in Eq.~\ref{DELTA-E}).


\item The noise on the solutions depend on the {\it total flux} in the
  beam.  Peeling\footnote{Solve for antenna based gains separately
    towards each direction of interest (DoI) (e.g. towards strong
    sources in the field) and use them to subtract the effects of the
    flux towards various DoI.} uses a fraction of the flux in the beam
  to solve for the DD parameters.  Solvers for parametrized
  measurement equation use all the flux in the beam for all the DD
  parameters.  Hence, DD calibration approaches based on Peeling will
  be limited by the available signal per Peeling solution.  This makes
  Peeling non-optimal from the point of view of throwing larger number
  of degree-of-freedom ($N_a*N_{DoI}$ where $N_{DoI}$ is the number of
  distinct directions-of-interest (DoI)) as well as reducing the
  available signal to noise ratio per solution.

\item Assuming we will necessarily need to use antenna based
  calibration algorithms (it can be argued that we do {\it
    necessarily} need to use such algorithms), and that the set of
  antenna based parameters $\{\vec{a}\}$ are independent, the
  achievable RMS noise in the image is
  $\sigma(\{\vec{a}\})/N_a$.

\end{enumerate}


%%&latex
%\documentclass[namedreferences]{SolarPhysics}
%%\usepackage[optionalrh]{spr-sola-addons} % For Solar Physics 
%%\documentclass[12pt]{article}
%\usepackage{graphicx}        % For eps figures, newer & more powerfull
%\usepackage{color}           % For color text: \color command
%\usepackage{url}             % For breaking URLs easily trough lines
%\def\UrlFont{\sf}            % define the fonts for the URLs
%
%\begin{document}
%\begin{article}
%
%\begin{opening}
%\title{The case for redundant arrays in Large-N designs?}
%\author{Divya Oberoi, 20 Nov, 2009}
%\end{opening}

\section{Basic Premise}
\label{S-intro}
\begin{enumerate}
\item The noise in the map, $\sigma_{image}$, is proportional to $1/\sqrt{N_{vis}}$, where $N_{vis} = N_{ant} \times (N_{ant-1})/2$. 
This leads to 
\begin{equation}
\sigma_{image} \approx1/N_{ant}
\end{equation}
Taking into account the time bandwidth product of the observations which are used to produce a single map, the above modifies to
\begin{equation}
\sigma_{image} \approx\frac{1}{N_{ant}} \times \frac{1}{\sqrt{\Delta t \Delta \nu}}
\end{equation}

\item On the other hand, the noise on estimation of antenna based parameters, $\sigma_{ant}$, is proportional to $N_{ant}^{-1/2}$.
Taking into the account the gains from the available time bandwidth product leads to.
\begin{equation}
\sigma_{ant} \approx\frac{1}{\sqrt{N_{ant}}} \times \frac{1}{\sqrt{\Delta \tau \Delta \nu^{\prime}}}
\end{equation}
where $\Delta \tau$ and $\Delta \nu^{\prime}$ represents the time and bandwidth interval, respectively, over which a solution for the antenna based parameters must be found.
It is reasonable to expect $\Delta \tau << \Delta t$ and $\Delta \nu^{\prime} << \Delta \nu$.

\item The question of interest now is to figure out how does the error in estimation of antenna based effects impact the noise in the image domain, which is addressed by Bhatnagar (Sep 13, 09).
He concludes that the achiveable noise in the image will be $\approx \sigma_{ant}/N_{ant}$.
\item Let's examine the ratio of expected noise in the image domain, based on propagating the erros in estimation of antenna parameters to the image, to the usual expectation of thermal noise in the image.
\begin{equation}
R = \sqrt{\frac{1}{N_{ant}} \frac{ \Delta t}{\Delta \tau} \frac{\Delta \nu}{\Delta \nu^{\prime}}}
\label{E-R-def}
\end{equation}
\begin{enumerate}
\item For the case of $N_{ant}$ = 100, an 8 hour observation with a calibration every 5 min and a 500 MHz bandwidth with a calibration step of 5 MHz, R evaluates to 10. 
Hence the noise will be dominated by an order of magnitude by $\sigma_{ant}$, or naively speaking one could have obtained this level of noise in a 100 times smaller time bandwidth product, a rather drastic consequence.
.
\item In the limiting case where $\Delta t = \Delta \tau$ and $\Delta \nu = \Delta \nu^{\prime}$, $R = N_{ant}^{-1/2}$, implying that the benefits from an increased number of antenna would bring shallower returns than expected.
\item It also implies that building stable hardware which does not require frequent calibration can lead to significant improvements in R. 
A similar argument can be made for spectral performance as well.
\end{enumerate}

\item For the case of a redundant array, where the fraction of independent visibilies generated by the array is parametrized by $\alpha$, the noise will 
have the dependence $1/(\alpha N_{ant})$. 
In such a case, the Eq. \ref{E-R-def} will be modified to
\begin{equation}
R_{\alpha} = \sqrt{\frac{\alpha}{N_{ant}} \frac{ \Delta t}{\Delta \tau} \frac{\Delta \nu}{\Delta \nu^{\prime}}}
\label{E-R1-def}
\end{equation}
In order to get R to come close to unity, one would need a very small value of $\alpha$, even a value of 0.1 will lead to an R of $\sqrt{10}$ in the example quoted earlier.
\end{enumerate}

This discussion suggests that we should expect noise in the image domain to be well beyond the expectations based on thermal noise if we continue to follow our approach for large N systems.
A desire to balance the noise contribution from the calibration errors of antenna parameters and thermal noise in the image plane by reducing the number of independent uv spacings sampled by an array, leads us towards building very heavily redundant arrays.

\section{Noise Charateristics}
\begin{enumerate}
\item {\bf Reproduced from Bhatnagar(13, Sep, 2009) to establish the same conventions} Assuming identical antennas, only additive random noise and only antenna based errors, the observed visbilities from a given baseline can be modelled as
\begin{equation}
V_{ij}^{Obs} = f({a_i}, I^M) + \sigma_{ij}
\end{equation}
where ${a_i}$ is the set of antenna based parameters, $I^M$ is the model of the sky emission and $f$ is a function that models the observed data.
$\sigma_{ij}$ is the additive random noise for the visibility and is related to observation parameters as
\begin{equation}
\sigma_{ij} = \frac {2 k_b T_{sys}} {\eta_a A \sqrt{\nu_{corr} \tau_{corr}}}
\end{equation}
\begin{equation}
= \frac {SEFD} {\sqrt{\nu_{corr} \tau_{corr}}}
\label{E-sigma_ij-def}
\end{equation}
where $\nu_{corr}$ and $\tau_{corr}$ are the correlator bandwidth and integration time, respectively.
The above equation assumes that the noise on data samples separated by $\nu_{corr}$ and $\tau_{corr}$ in frequency and time, respectively, is statistically independent.
$\sigma_{ij}$ is gaussian random noise of thermal origin and is not correlated across baselines ot antenna.
\item Let $g_i^{Mod}$ be antenna dependent complex multiplicative gains, estimated from the data. Then
\begin{equation}
g_i^{Mod} = g_{i}^{Obs} + \sigma(g_i)
\end{equation}
where $g_i^{Obs}$ is the true value of this gain and $\sigma(g_i)$ is the error associated with the estimation process and is related to the dataset as follows
\begin{equation}
\sigma(g_i) = \frac {\sigma_{ij}} {S \bar{g}} \frac {1} {\sqrt{N_a \nu_{sol} \tau_{sol}}}
\end{equation}
where $S$ is the total flux in the beam, $\bar{g}$ is the average antenna based gain over a solution bandwidth and time span of $\nu_{sol}$ and $\tau_{sol}$, respectively.
Substituting from Equation \ref{E-sigma_ij-def} leads to
\begin{equation}
\sigma(g_i) = \frac {SEFD} {\sqrt{\nu_{corr} \tau_{corr}}} \frac {1} {S \bar{g}} \frac {1} {\sqrt{N_a \nu_{sol} \tau_{sol}}}
\end{equation}
where $\nu_{corr} \le \nu_{sol}$ and $\tau_{corr} \le \tau_{sol}$ must hold, though usually $\nu_{corr} << \nu_{sol}$ and $\tau_{corr} << \tau_{sol}$.

Eq.~\ref{DISigma}\footnote{I suppose you meant this equation instead
  of ``Eq. 11''?  I can't see what's wrong in this equation.  On the
  other hand $N_a, N_\nu and N_\tau$ in Eq.~\ref{DISigmaDiv} would
  leave a factor $\sqrt{\\nu_{corr}\tau_{corr}}$ unaccounted for.} is not correct the way it currently stands, it needs a minor fix, it should really be
\begin{equation}
\label{DISigmaDiv}
\sigma(g_i) = \frac {SEFD} {\sqrt{\nu_{corr} \tau_{corr}}} \frac {1} {S \bar{g}} \frac {1} {\sqrt{N_a N_{\nu} N_{\tau}}}
\end{equation}
where $N_{\nu} = \nu_{sol}/\nu_{corr}$ and $N_{\tau} = \tau_{sol}/\tau_{corr}$ 
\item 
For the special case of only one point source in the sky, the visbilities can be modeled as
\begin{equation}
V_{ij}^{Obs} = g_i^{Obs} g_j^{Obs} S + \sigma_{ij}
\end{equation}
Expressing the model visibilities in terms of model gains for the special case of a single point source of flux $S Jy$ in the beam leads to
\begin{equation}
V_{ij}^{Mod} = g_i^{Mod} g_j^{Mod} S 
\end{equation}
\begin{equation}
= (g_i^{Obs} + \sigma(g_i)) (g_j^{Obs} + \sigma(g_j)) S
\end{equation}
\begin{equation}
= (g_i^{Obs} g_j^{Obs} + g_i^{Obs} \sigma(g_j) +  g_j^{Obs} \sigma(g_i)) S
\end{equation}
ignoring the second order terms in $\sigma(g_i)$.
The residuals between the observed and modelled visibilities are given by
\begin{equation}
V_{ij}^{Obs} - V_{ij}^{Mod} = (g_i^{Obs}\sigma(g_j) + g_j^{Obs}\sigma(g_i)) S + \sigma_{ij}
\end{equation}
\begin{equation}
\approx 2\bar g^{Obs}\bar \sigma(g_j) S + \sigma_{ij}
\end{equation}
where $\bar g^{Obs}$ is the average observed gain and $\sigma(g_j)$, the average uncertainty on the observed gain.
Substituting from the expressions for $\sigma(g_j)$ and $\sigma_{ij}$ leads to
\begin{equation}
V_{ij}^{Obs} - V_{ij}^{Mod} \approx
2
\frac {SEFD} {\sqrt{\nu_{corr} \tau_{corr}}} \frac {1} {\sqrt{N_a N_{\nu} N_{\tau}}} + \frac {SEFD} {\sqrt{\nu_{corr} \tau_{corr}}}
\end{equation}
These residuals should define the noise floor for the observations.
Seems from here the contribution of the errors in gain estimates is always bound to be smaller than the contribution of thermal noise by at least a factor of $2/\sqrt{N_a}$ and usually more depending upon the magnitude of $\sqrt{N_{\nu} N_{\tau}}$.
\end{enumerate}

%\end{article}
%\end{document}

\end{document}

For the sake of understanding, let us further assume that $X_{ij}=1.0$
(the ideal nominal "point source" visibilities), $g_j$s and
$\sigma_{ij}$ are independent gaussian random variates.  The RMS noise
on $g_i$ can be estimated as
\begin{eqnarray}
\sigma(g_i) &=& \sqrt{\frac{C(g_i)}{N_a}}\\
&\propto& \frac{SEFD}{\sqrt{N_a \Delta \nu_{sol}\tau_{sol}}}
\end{eqnarray}

For an observation of a point source at the phase center, imaging
operation is just the vector sum $\sum_{ij}\frac{V^{Obs}_{ij}}{g_i
  g^*_j}$ where $g_i$s represent the solutions after iterations in
Eq.~\ref{LSQ} have converged.


The equation describing the iterative antenna based complex gain
solver to solve for $g_i$s, given $V^{Obs}_{ij}$ and $V^M_{ij}$ can be
written as
\begin{equation}
\label{LSQ}
\displaystyle g^N_i = g^{N-1}_i + \alpha\frac{\sum^{N_a}_j g^*_j
  X_{ij}}{\sum^{N_a}_j g^*_j} 
\end{equation}
where $g^N_i$ is the solution at iteration $N$, $N_a$ is the total
number of antennas in the array and $\alpha$ is the loop gain
($\alpha<1.0$). $X_{ij} = V^{Obs}_{ij}/V^M_{ij}$ and $n_{ij} =
\frac{1}{V^M_{ij}}\frac{\sigma_{ij}}{\sqrt{\Delta \nu_{sol}
    \tau_{sol}}}$, where $\Delta \nu_{sol}$ and $\tau_{sol}$ are the
solution bandwidth and time interval over which $g_i$s are assumed to
be constant.

