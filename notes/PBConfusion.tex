\documentclass[11pt,epsf]{article}
%\documentstyle[12pt,amsmath]{article}
%\usepackage{html}
\usepackage{epsf}
\usepackage{amsmath}
\usepackage[dvips]{graphicx, color}  % The figure package

%\setlength{\textheight}{23.0cm}
%\setlength{\textwidth}{15.00cm}
%\setlength{\topmargin}{-1.5cm}
%\setlength{\oddsidemargin}{1.5cm}
%\setlength{\evensidemargin}{1.5cm}
%\setlength{\parskip}{5pt}
%\setlength{\parindent}{20pt}

%\evensidemargin -0.7cm
%\oddsidemargin 1.5cm
%\textwidth 15cm
%\topmargin -1.5cm
%\textheight 23cm
\parskip 1ex    % White space between paragraphs amount

\begin{document}
\title{Documenting the PB-Confusion in CASA Synthesis Library}
\author{S. Bhatnagar}
\date{Jan. 09, 2009}
\maketitle

\normalsize

%%%%%%%%%%%%%%%%%%%%%%%
\begin{abstract}
%%%%%%%%%%%%%%%%%%%%%%%
Here is my take on the sequences of operations required for FFT based
forward and reverse transforms used in iterative deconvolution
algorithms.
\end{abstract}

\section{The measurement process}

The antenna surface acts a focusing (non-imaging) device, which
reflects all the radiation incident on its surface to a single point
in space (the location of the feed).  The measurements from the feeds
of various antennas are then correlated\footnote{I am not deriving
this part here - but I think we should get this also from first
principles, just to be sure.} at the correlator to compute the
visibility, $V_{ij}$, measured by two antennas denoted by sub-script
$i$ and $j$.

For the purpose of analysis, we can imagine a ``visibility field'' due
to the spatially incoherent emission at infinity pervading all space
in the measurement plane (the plane of {\it all} antennas).  Lets
denote it by $V^o$.  An interferometric telescope can then be
described as an array of ``collecting'' devices, the electromagnetic
description of which can be given by $A^o=A^o_i \star
A^{o^\star}_j$ where $A_i$ is the aperture illumination function of
the antennas.  Measurement from a baseline of the interferometer can
then be written as
\begin{equation}
 V_{ij}^{Obs} = (A^o \star V^o) \cdot S_{ij}
\end{equation}
where $S_{ij}$ is the uv-sampling function.  Each measurement
$V_{ij}^{Obs}$ is also assigned a weight, $W_{ij}$, which is inversely
proportional to the {\it measured} system temperature denoted by~
$T_{ij}^{sys}$.

\section{Forward transform (Vis-to-Image domain)}

To grid the measured data on to a regular grid, lets construct a
convolution function given by $A^M_{ij}$.  The process of re-sampling
the data on a regular grid can then be written as
\begin{eqnarray}
\label{ME}
V^G &+=& A^M_{ij} \star \left[V^{obs}_{ij} \cdot W_{ij} \cdot S_{ij}\right]\\
&+=&A^M_{ij} \star \left[\left(A^o_{ij}\star V^o_{ij}\right) \cdot  \left(S_{ij}\cdot W_{ij}\right)\right]
\end{eqnarray}
Note that this operation ``sprays'' each measured visibility to a
number of grid points, weighted by the function $A^M_{ij}$ (hence the
term ``projection'' associated with such transforms).

The image is then computed as 
\begin{eqnarray}
I^{D^M} &=& FT \left[V^G\right]\\
\label{DIRTY_IMAGE_COMP}
        &=& PB^M \cdot \left[\left(PB^o\cdot I^o\right)\star PSF\right]
\end{eqnarray}
Since Fourier Transform is essentially a vector average of its
arguments, the effects of $PB$ in Eq.~\ref{DIRTY_IMAGE_COMP} can be
interpreted as ``dirty image accumulating $PB^M\cdot PM^o$.  It is
less clear in Eq.~\ref{DIRTY_IMAGE_COMP} why
$avg\left[PB_i^{M^2}\right]$ is a better estimate of $PB^M \cdot
PM^o$.  It is clearer from Eq.~\ref{ME} - once we recognize that
algorithms that correct for direction dependent effects in the
visibility plane can converge {\it only} if $A^M_{ij} \star A^o_{ij}$
is close to 1.0 (basically the argument in our paper, that the
equivalent operators must be unitary (or approximately so)).

Current implementation of MS-MFS uses MS-Clean in minor cycle.  That
being essentially Hogbom Clean, {\it requires} that the Dirty Image be
accurate.  However since in MS-MFS, which does reconcile with the
data, this should only be a matter of rate of convergence rather than
accuracy of convergence.

Convolution function corrected image is then computed as
\begin{eqnarray}
\label{DIRTY_IMAGE}
I^{D} &=& \frac{I^{D^M}}{PB^M}\\
where~ PB^M &=& FT\left[\frac{\sum_{ij} \left( A^M_{ij} \star A^o_{ij} \right) W_{ij}}{\sum_{ij} W_{ij}}\right]
\end{eqnarray}
Area under $A^M_{ij}$ must be unity.  Since $A^M_{ij}$ is the {\it
best} available model for antenna aperture illumination, in practice
$A^o_{ij}$ is replaced by $A^M_{ij}$ in the above equation.  The PSF
is similarly computed with $V^{Obs}_{ij}$ set to 1.0.

\section{Conclusions}
Following is how I interpret these equations:
\begin{enumerate}
\item The PSF is is ``flat''.  I.e, it is {\it not} multiplied by PB.
\item $I^D$ is convolved with this ``flat PSF''.  True sky ($I^o$) in $I^D$
  is multiplied by the true PB ($PB^o$).  
\end{enumerate}
The following then appears to be an appropriate prescription for
iterative deconvolution
\begin{enumerate}
\item Compute the Dirty Image and the PSF using Eq.~\ref{DIRTY_IMAGE}
\item Do the minor cycle using these images and compute the model
  image $I^M$.
\item Use $I^M/PB^M$ to predict the visibilities in the major cycle
\end{enumerate}

The deconvolved image will be ``true sky'' multiplied by the PB.  That
can then be corrected for PB (if required) using $PB^M$ (the best
estimate of the average PB, given the model for the antenna aperture
illumination function $A^M$).

\section{Acknowledgments}
Is this correct?  If not, what am I missing?  The issue as I see them,
for deconvolution algorithms, is Eq.~\ref{ME} correct in the ordering
of terms.  My take is - the ordering is determined by the order in
which operations are done (physical measurements as well as
computational operations).

These are the precise equations I have used in the PBWProjectFT.  It
works down to thermal limit on the IC2233 L-Band data (Stokes-I and
-V).  I also simulated data using NVSS field with sources all over the
PB.  Looking at the sources in the Dirty Image close to the edge of
the PB confirmed (from what I can recall anyway) the conclusions.  I
should (and will) simulate again and attach those images with this note
to clarify a bit.

\end{document}
