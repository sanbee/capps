\documentclass[11pt,epsf]{article}
%\documentstyle[12pt,amsmath]{article}
%\usepackage{html}
\usepackage{epsf}
\usepackage{amsmath}
\usepackage[dvips]{graphicx, color}  % The figure package

%\setlength{\textheight}{23.0cm}
%\setlength{\textwidth}{15.00cm}
%\setlength{\topmargin}{-1.5cm}
%\setlength{\oddsidemargin}{1.5cm}
%\setlength{\evensidemargin}{1.5cm}
%\setlength{\parskip}{5pt}
%\setlength{\parindent}{20pt}

%\evensidemargin -0.7cm
%\oddsidemargin 1.5cm
%\textwidth 15cm
%\topmargin -1.5cm
%\textheight 23cm
\parskip 1ex    % White space between paragraphs amount

\begin{document}
\title{Documenting the PolConfusion in AIPS++ Synthesis Library}
\author{S. Bhatnagar}
\date{June 23, 2006\\\small{Updated: Dec. 8, 2009}}
\maketitle

\normalsize

%%%%%%%%%%%%%%%%%%%%%%%
\begin{abstract}
%%%%%%%%%%%%%%%%%%%%%%% This document documents the amusing
  (irritating for some) comedy of indices and complex conjugation that
  has to be handled when gridding and de-gridding visibilities using
  the {\tt pbwproject/pbmosaic} FTMachine in AIPS++.  This FTMachine
  corrects for three terms of the Measurement Equation: (1) rotation
  of the polarized Primary Beams, (2) the w-term, and (3) antenna
  pointing errors.  These terms need to be complex conjugated
  differently for gridding the visibilities onto a regular grid, for
  predicting the visibilities from a regular grid and depends on the
  sign of the $w$ co-ordinate, leading to the {\it PolConfusion}
  described below.  Enjoy.
\end{abstract}

\section{The {\tt pbwproject/pbmosaic} FT Machine}

The {\tt pbwprojec} FT Machine in AIPS++ corrects for 
\begin{enumerate}
\item rotation of the polarized Primary Beams on the sky with
  Parallactic Angle
\item the w-term, and
\item antenna pointing errors
\end{enumerate}

In the following, imaging of and visibility prediction for only {\tt
  RR} and {\tt LL} planes is described.  Convolution functions for all
polarization planes are computed as
\begin{eqnarray}
C_{RR} = E_{RR} \star W \nonumber\\
C_{LL} = E_{LL} \star W
\end{eqnarray}
where $E$ is the aperture plane filter and $W$ is the w-term filter.

The polarization plane order for the visibilities (MS), sky image
($I^{Sky}$), and the convolution function ($C$) is given in Table~\ref{TAB:POLORDER}.
\begin{table}[!ht]
\caption{PolOrder for the various objects involved in imaging}
\label{TAB:POLORDER}
\begin{center}
\begin{tabular}{|l|c|c|}
\hline
         & Plane 0 & Plane 1\\
\hline
MS       &  RR     &  LL\\
$C$      &  RR     & LL\\
$I^{Sky}$&  LL     &  RR\\
\hline
\end{tabular}
\end{center}
\end{table}
The {\tt FTMachine} object holds an array which maps the visibility
polarization axis to the imaging polarization axis.  This map, for the
above case is {\tt PolMap = [1,0]} - i.e. the $i^{th}$ polarization
plane of the MS ({\tt MS[i]}) maps to the {\tt PolMap[i]} plane of the
image ($MS[i] \rightarrow I[PolMap[i]]$).

\section{Prediction (De-Gridding)}

\subsection{Mathematical requirement}

Mathematically speaking (though the following involves the sign
convention for $FFT$ and $FFT^{-1}$), the following is what is
required:

\begin{eqnarray}
ConvFunc_{RR} &=& W \star E_{RR}~for~w > 0 \nonumber \\
ConvFunc_{RR} &=& W^* \star E_{RR}~for~w <= 0 
\end{eqnarray}

\subsection{Implementation in the code (\tt fpbmos.f)}

{\bf Warning: Note that in the code, the indices for $C$ derived from
  {\tt PolPlane/ConjPlane} maps value are reversed.  I.e. value that
  should be used from PolPlane is used from ConjPlane and vice
  versa.   The index used for $V^G$ derived from {\tt PolMap} is
  correct in the code as well.}

To apply the w-term correctly, along with the $E$ term, the equations
for predicting the visibilities ($V$) from a regular complex grid
($V^G$) for polarization plane {\tt RR} are:
\begin{eqnarray}
 V_{RR} &=& E_{RR} \star W \star V^G_{RR}~for~w>0  \nonumber \\
 V_{RR} &=& E_{RR} \star W^* \star V^G_{RR}~for~w<=0 
\end{eqnarray}
The visibility polarization planes are picked up from the {\tt
  VisBuffer} serially - i.e. the loop over all polarizations goes from
0 to $N_{pol}$.  Therefore, given the order for the MS in
Table~\ref{TAB:POLORDER}, visibilities are predicted in the order {\tt
  RR, LL}.  The order in which the gridded visibility planes are
treated follow the order of $I^{Sky}$.  The correct mapping of $V^G$
below is therefore in {\tt PolMap}.  Noting that $C_{RR}^* = C_{LL}$,
the above equations can be realized by keeping another map {\tt
  ConjMap=[0,1]} and de-gridding as :
\begin{eqnarray}
 V[i] &=& C[ConjMap[i]] \star V^G_{RR}[PolMap[i]]~for~w>0 \nonumber \\
 V[i] &=& C[PolMap[i]]^* \star V^G_{RR}[PolMap[i]]~for~w<=0 
\end{eqnarray}
This is equivalent to
\begin{eqnarray}
 MS[0]&=&C[0] \star V^G[1]~for~w>0\nonumber\\
 MS[0]&=&C[1]^* \star V^G[1]~for~w<=0
\end{eqnarray}
which is 
\begin{eqnarray}
 MS[RR]&=&C[RR] \star V^G[RR]~for~w>0\nonumber\\
 MS[RR]&=&C[LL]^* \star V^G[RR]~for~w<=0
\end{eqnarray}







\section{Gridding}
\subsection{Mathematical requirement}

Mathematically speaking (though the following involves the sign
convention for $FFT$ and $FFT^{-1}$), the following is what is
required:

\begin{eqnarray}
ConvFunc_{RR} &=& W^* \star E^*_{RR}~for~w > 0 \nonumber \\
ConvFunc_{RR} &=& W \star E^*_{RR}~for~w <= 0 
\end{eqnarray}

\subsection{Implementation in the code (\tt fpbmos.f)}
{\bf Warning: Note that in the code, the indices for $C$ derived from
  {\tt PolPlane/ConjPlane} maps value are reversed.  I.e. value that
  should be used from PolPlane is used from ConjPlane and vice
  versa.  The index used for $V^G$ derived from {\tt PolMap} is
  correct in the code as well.}

For imaging, the gridded visibilities are computed as:
\begin{eqnarray}
 V^G_{RR} &=& E_{RR}^* \star W^* \star V_{RR}~for~w>0 \nonumber \\
 V^G_{RR} &=& E_{RR}^* \star W \star V_{RR}~for~w<=0
\end{eqnarray}
This is equivalent to
\begin{eqnarray}
 V^G_{RR} &=& E_{LL} \star W^* \star V_{RR}~for~w>0 \nonumber \\
 V^G_{RR} &=& E_{LL} \star W \star V_{RR}~for~w<=0
\end{eqnarray}
Again, the polarization planes from $V^G$ are picked up serially, and
hence gridding gets done in the order {\tt LL,RR}.
Given the mapping as in Table~\ref{TAB:POLORDER}, this can be realized
as:
\begin{eqnarray}
 V^G[PolMap[i]] &=& C[PolMap[i]]^* \star V_{RR}[i]~for~w>0  \nonumber \\
 V^G[PolMap[i]] &=& C[ConjMap[i]] \star V_{RR}[i]~for~w<=0
\end{eqnarray}
This is equivalent to
\begin{eqnarray}
 V^G[1]&=&C[1]^* \star MS[0]~for~w>0\nonumber\\
 V^G[1]&=&C[0] \star MS[0]~for~w<=0
\end{eqnarray}
which is 
\begin{eqnarray}
 V^G[RR]&=&C[RR]^* \star MS[RR]~for~w>0\nonumber\\
 V^G[RR]&=&C[LL] \star MS[RR]~for~w<=0
\end{eqnarray}


\end{document}
