%%%%%%%%%%%%%%%%%%%%%%%%%%%%%%%%%%%%%%%%%%%%%%%%%%%%%%%%%%%%%%%%%
%%%%%%%%%%%%%%%%%%%%%%%%%%%%%%%%%%%%%%%%%%%%%%%%%%%%%%%%%%%%%%%%%
%%%%%%%%%%%%%%%%%%%%%%%%%%%%%%%%%%%%%%%%%%%%%%%%%%%%%%%%%%%%%%%%%

% STM 2007-04-13  split from previous version
% STM 2007-08-01  add immoments and regridimage
% STM 2007-10-10  beta release (spell-checked)

\chapter{Image Analysis}
\label{chapter:analysis}

\begin{wrapfigure}{r}{2.5in}
  \begin{boxedminipage}{2.5in}
     \centerline{\bf Inside the Toolkit:}
     Image analysis is handled in the {\tt ia} tool.
     Many options exist there, including region statistics and
     image math. See \S~\ref{section:analysis.toolkit} below
     for more information.
  \end{boxedminipage}
\end{wrapfigure}

Once data has been calibrated (and imaged in the case of synthesis
data), the resulting image or image cube must be displayed or 
analyzed in order to extract quantitative information, such as
statistics or moment images.  In addition, there need to be facilities for
the coordinate conversion of images for direct comparison.

{\bf BETA ALERT:} We have assembled a skeleton of image analysis tasks for
this release.  Many more are still under development.

The image analysis tasks are:
\begin{itemize}
   \item {\tt imhead} --- summarize and manipulate the ``header'' 
         information in a CASA image 
         (\S~\ref{section:analysis.imhead})
   \item {\tt immoments} --- compute the moments of an image cube
         (\S~\ref{section:analysis.moments})
   \item {\tt regridimage} --- regrid an image onto the coordinate
         system of another image 
         (\S~\ref{section:analysis.regrid})
   \item {\tt importfits} --- import a FITS image into a CASA  
         {\it image} format table 
         (\S~\ref{section:analysis.fits})
   \item {\tt exportfits} --- write out an image in FITS format
         (\S~\ref{section:analysis.fits})
\end{itemize}

There are other tasks which are useful during image analysis.  These
include:
\begin{itemize}
   \item {\tt viewer} --- there are useful region statistics and
         image cube slice and profile capabilities in the viewer 
         (\S~\ref{chapter:display})
\end{itemize}

We also give some examples of using the CASA Toolkit to aid in
image analysis (\S~\ref{section:analysis.toolkit}).

%%%%%%%%%%%%%%%%%%%%%%%%%%%%%%%%%%%%%%%%%%%%%%%%%%%%%%%%%%%%%%%%%
%%%%%%%%%%%%%%%%%%%%%%%%%%%%%%%%%%%%%%%%%%%%%%%%%%%%%%%%%%%%%%%%%
\section{Summary of an Image and Headers}
\label{section:analysis.imhead}

To summarize and change keywords and values in the ``header'' of
your image, use the {\tt imhead} task.  Its inputs are:
\small
\begin{verbatim}
#  imhead :: Lists/gets/puts/stats of image header properties:

imagename   =         ''   #   Name of input image file
mode        =  'summary'   #   Options: get, put, summary, list, stats
\end{verbatim}
\normalsize

The default is {\tt mode='summary'}, which will print out a summary,
for example
\small
\begin{verbatim}
  imhead('ngc5921.usecase.clean.image','summary')
\end{verbatim}
\normalsize
prints in the logger: 
\small
\begin{verbatim}
Opened image ngc5921.usecase.clean.image

Image name       : ngc5921.usecase.clean.image
Object name      : 
Image type       : PagedImage
Image quantity   : Intensity
Pixel mask(s)    : None
Region(s)        : None
Image units      : Jy/beam
Restoring Beam   : 51.5254 arcsec, 45.5987 arcsec, 14.6417 deg

Direction reference : J2000
Spectral  reference : LSRK
Velocity  type      : RADIO
Rest frequency      : 1.42041e+09 Hz
Pointing center     :  15:22:00.000000  +05.04.00.000000
Telescope           : VLA
Observer            : TEST
Date observation    : 1995/04/13/00:00:00

Axis Coord Type      Name             Proj Shape Tile   Coord value at pixel    Coord incr Units
------------------------------------------------------------------------------------------------ 
0    0     Direction Right Ascension   SIN   256   64  15:22:00.000   128.00 -1.500000e+01 arcsec
1    0     Direction Declination       SIN   256   64 +05.04.00.000   128.00  1.500000e+01 arcsec
2    1     Stokes    Stokes                    1    1             I
3    2     Spectral  Frequency                46    8   1.41281e+09     0.00  2.441406e+04 Hz
                     Velocity                               1603.56     0.00 -5.152860e+00 km/s
\end{verbatim}
\normalsize

If you choose {\tt mode='list'}, you get the summary in the logger and
a listing of keywords and values to the terminal:
\small
\begin{verbatim}
Available header items to modify:

General --
         -- object  
         -- telescope VLA
         -- observer TEST
         -- epoch "1995/04/13/00:00:00"
         -- restfrequency "1420405752.0Hz"
         -- projection "SIN"
         -- bunit Jy/beam
         -- beam ["51.5253715515arcsec","45.5987281799arcsec","14.6416902542deg"]
axes --
         -- ctype1 Right Ascension
         -- ctype2 Declination
         -- ctype3 Stokes
         -- ctype4 Frequency
crpix --
         -- crpix1 128.0
         -- crpix2 128.0
         -- crpix3 0.0
         -- crpix4 0.0
crval --
         -- crval1 4.02298392585
         -- crval2 0.0884300154344
         -- crval3 1.0
         -- crval4 1412808153.26
cdelt --
         -- cdelt1 -7.27220521664e-05
         -- cdelt2 7.27220521664e-05
         -- cdelt3 1.0
         -- cdelt4 24414.0625
units --
         -- cunit1 rad
         -- cunit2 rad
         -- cunit3 
         -- cunit4 Hz

\end{verbatim}
\normalsize

The values for these keywords can be queried using
{\tt mode='get'}.  This opens sub-parameters
\small
\begin{verbatim}
mode           =      'get'   #   Options: get, put, summary, list, stats
     hditem    =         ''   #   header item to get or put
\end{verbatim}
\normalsize
You can set the values for these keywords using
{\tt mode='put'}.  This opens sub-parameters
\small
\begin{verbatim}
mode           =      'put'   #   Options: get, put, summary, list, stats
     hditem    =         ''   #   header item to get or put
     hdvalue   =         ''   #   header value to set (for mode=put)
\end{verbatim}
\normalsize

For example, continuing the above example:
\small
\begin{verbatim}
CASA <6>: mode = 'get'
CASA <7>: hditem = 'observer'
CASA <8>: imhead()
***
observer ::  TEST
  Out[8]: {'observer': 'TEST'}

CASA <9>: mode = 'put'
CASA <10>: hdvalue = 'CASA'
CASA <11>: imhead()
  Out[11]: {'observer': 'CASA'}

CASA <12>: mode = 'list'
CASA <13>: imhead()
Available header items to modify:

General --
         -- object  
         -- telescope VLA
         -- observer CASA
...
\end{verbatim}
\normalsize
Note that these options return a Python
dictionary containing the current value of the {\tt hditem}
(after updating in the case of {\tt mode='put'}).  This
dictionary can be manipulated in Python in the usual manner.
For example, continuing from the above,
\small
\begin{verbatim}
CASA <14>: mode = 'get'
CASA <15>: hditem = 'observer'
CASA <16>: myobs = imhead()
***
observer ::  CASA

CASA <17>: print myobs
{'observer': 'CASA'}
CASA <18>: print myobs['observer']
CASA

CASA <19>: mode = 'put'
CASA <20>: hdvalue = 'VLA'
CASA <21>: newobs = imhead()

CASA <22>: print newobs
{'observer': 'VLA'}

\end{verbatim}
\normalsize

Finally, {\tt mode = 'stats'} can be used to obtain statistics over
the whole image, again returned as a dictionary.  For example,
\small
\begin{verbatim}
CASA <23>: mode = 'stats'
CASA <24>: imhead()
  Out[24]: 
{'blc': array([0, 0, 0, 0]),
 'blcf': '15:24:08.404, +04.31.59.181, I, 1.41281e+09Hz',
 'flux': array([ 4.07744818]),
 'max': array([ 0.05234427]),
 'maxpos': array([134, 134,   0,  38]),
 'maxposf': '15:21:53.976, +05.05.29.998, I, 1.41374e+09Hz',
 'mean': array([  1.60033267e-05]),
 'min': array([-0.01049265]),
 'minpos': array([160,   1,   0,  30]),
 'minposf': '15:21:27.899, +04.32.14.923, I, 1.41354e+09Hz',
 'npts': array([ 3014656.]),
 'rms': array([ 0.00202178]),
 'sigma': array([ 0.00202172]),
 'sum': array([ 48.24452486]),
 'sumsq': array([ 12.32271508]),
 'trc': array([255, 255,   0,  45]),
 'trcf': '15:19:52.390, +05.35.44.246, I, 1.41391e+09Hz'}

CASA <25>: mystats = imhead()

CASA <26>: print mystats['rms']
[ 0.00202178]

\end{verbatim}
\normalsize
{\bf TOOLKIT NOTE:} The {\tt mode='stats'} option in the {\tt imhead}
task uses the {\tt ia.summary} and {\tt ia.statistics} method
(with a mask set to the entire image) --- 
see \S~\ref{section:analysis.toolkit} below.

%%%%%%%%%%%%%%%%%%%%%%%%%%%%%%%%%%%%%%%%%%%%%%%%%%%%%%%%%%%%%%%%%
%%%%%%%%%%%%%%%%%%%%%%%%%%%%%%%%%%%%%%%%%%%%%%%%%%%%%%%%%%%%%%%%%
\section{Computing the Moments of an Image Cube ({\tt immoments})}
\label{section:analysis.moments}

For spectral line datasets, the output of the imaging process is an
{\tt image cube}, with a frequency or velocity channel axis in
addition to the two sky coordinate axes.  This can be most easily
thought of as a series of image {\tt planes} stacked along the
spectral dimension.

A useful product to compute is to collapse the cube into a 
{\it moment} image by taking a linear combination of the individual
planes:
\begin{equation}
   M_m(x_i,y_i) = \sum_k^N w_m(x_i,y_i,v_k)\,I(x_i,y_i,v_k)
\end{equation}
for pixel $i$ and channel $k$ in the cube $I$.  There are a number
of choices to form the $m$ moment, usually approximating some
polynomial expansion of the intensity distribution over velocity
mean or sum, gradient, dispersion, skew, kurtosis, etc.).  There
are other possibilities (other than a weighted sum) for calculating
the image, such as median filtering, finding minima or maxima along
the spectral axis, or absolute mean deviations.  And the axis along
which to do these calculation need not be the spectral axis (ie.
do moments along Dec for a RA-Velocity image).  We will treat all
of these as generalized instances of a ``moment'' map.

The {\tt immoments} task will compute basic moment images from a cube.
The default inputs are:
\small
\begin{verbatim}
#  immoments :: Compute moments of an image cube:

imagename    =         ''   #   Input image name
moments      =        [0]   #   List of moments to compute
axis         =          3   #   Axis for moment calculation
planes       =         ''   #   Set of planes/channels to use for moment(e.g,"3~20,21")
includepix   =       [-1]   #   Range of pixel values to include
excludepix   =       [-1]   #   Range of pixel values to exclude
outfile      =         ''   #   Output image file name (or root for multiple moments)
async        =      False   #   if True, run in the background, prompt is freed
\end{verbatim}
\normalsize

The choices for the operation mode are:
\small
\begin{verbatim}
    moments=-1  - mean value of the spectrum
    moments=0   - integrated value of the spectrum
    moments=1   - intensity weighted coordinate;traditionally used to get 
                  'velocity fields'
    moments=2   - intensity weighted dispersion of the coordinate; traditionally
                  used to get 'velocity dispersion'
    moments=3   - median of I
    moments=4   - median coordinate
    moments=5   - standard deviation about the mean of the spectrum
    moments=6   - root mean square of the spectrum
    moments=7   - absolute mean deviation of the spectrum
    moments=8   - maximum value of the spectrum
    moments=9   - coordinate of the maximum value of the spectrum
    moments=10  - minimum value of the spectrum
    moments=11  - coordinate of the minimum value of the spectrum
\end{verbatim}
\normalsize
The meaning of these is described in the CASA Reference Manual
(\url{http://casa.nrao.edu/docs/casaref/image.moments.html}).

For example, using the NGC5921 example (\S~\ref{section:scripts.ngc5921}):
\small
\begin{verbatim}
default('immoments')

imagename = 'ngc5921.usecase.clean.image'

# Do first and second moments
moments = [0,1]

# Need to mask out noisy pixels, currently done
# using hard global limits
excludepix = [-100,0.009]

# Include all planes
planes = ''

# Output root name
outfile = 'ngc5921.usecase.moments'

immoments()

# It will have made the images:
# --------------------------------------
# ngc5921.usecase.moments.integrated
# ngc5921.usecase.moments.weighted_coord

\end{verbatim}
\normalsize

In order to make a usable moment image, it is critical to set a
reasonable threshold to exclude noise from being added to the
moment maps.  Something like a few times the rms noise level
in the usable planes seems to work (put into {\tt includepix}
or {\tt excludepix} as needed.  Also use {\tt planes} to ignore
channels with bad data.

{\bf BETA ALERT:} We are working on improving the thresholding
of planes beyond the global cutoffs in {\tt includepix}
and {\tt excludepix}.

%%%%%%%%%%%%%%%%%%%%%%%%%%%%%%%%%%%%%%%%%%%%%%%%%%%%%%%%%%%%%%%%%
%%%%%%%%%%%%%%%%%%%%%%%%%%%%%%%%%%%%%%%%%%%%%%%%%%%%%%%%%%%%%%%%%
\section{Regridding an Image ({\tt regridimage})}
\label{section:analysis.regrid}

\begin{wrapfigure}{r}{2.5in}
  \begin{boxedminipage}{2.5in}
     \centerline{\bf Inside the Toolkit:}
     More complex coordinate system and image regridding 
     operation can be carried out in the toolkit.  The 
     {\tt coordsys} ({\tt cs}) tool and the {\tt ia.regrid}
     method are the relevant components.
  \end{boxedminipage}
\end{wrapfigure}

It is occasionally necessary to regrid an image onto a new coordinate
system.  The {\tt regridimage} task will regrid one image onto the
coordinate system of another, creating an output image.  In this
task, the user need only specify the names of the input, template, and
output images.  

If the user needs to do more complex operations, such as regridding an
image onto an arbitrary (but known) coordinate system, changing from
Equatorial to Galactic coordinates, or precessing Equinoxes, the CASA
toolkit can be used (see sidebox).  Some of these facilities will
eventually be provided in task form.

The default inputs are:
\small
\begin{verbatim}
#  regridimage :: Regrid imagename to have template image parameters

imagename   =         ''   #   Name of image to be regridded
template    =         ''   #   image having the parameters that is wanted in regridded image
output      =         ''   #   Name of image in which result of regridding is stored
async       =      False   #   if True run in the background, prompt is freed
\end{verbatim}
\normalsize

%%%%%%%%%%%%%%%%%%%%%%%%%%%%%%%%%%%%%%%%%%%%%%%%%%%%%%%%%%%%%%%%%
%%%%%%%%%%%%%%%%%%%%%%%%%%%%%%%%%%%%%%%%%%%%%%%%%%%%%%%%%%%%%%%%%
\section{Image Import/Export to FITS}
\label{section:analysis.fits}

To export your images to fits format use the {\tt exportfits} task.
The inputs are:
\small
\begin{verbatim}
#  exportfits :: Convert a CASA image to a FITS file

imagename    =         ''   #   Name of input CASA image
fitsimage    =         ''   #   Name of output FITS image
velocity     =      False   #   Prefer velocity for spectral axis
optical      =       True   #   Prefer optical velocity definition
bitpix       =        -32   #   Bits per pixel (-32 (floating point), 16 (integer))
minpix       =          0   #   Minimum pixel value
maxpix       =          0   #   Maximum pixel value
overwrite    =      False   #   Overwrite pre-existing output file
dropdeg      =      False   #   Drop degenerate axes
deglast      =      False   #   Put degenerate axes last in header
async        =      False   #   if True run in the background, prompt is freed

\end{verbatim}
\normalsize
For example,
\small
\begin{verbatim}
   exportfits('ngc5921.usecase.clean.image','ngc5921.usecase.image.fits')
\end{verbatim}
\normalsize

You can also use the {\tt importfits} task to import a FITS image into
CASA image table format.  Note, the CASA {\tt viewer} can read fits
images so you don't need to do this if you just want to look a the image.  
The inputs for {\tt importfits} are:
\small
\begin{verbatim}
#  importfits :: Convert an image FITS file into a CASA image:

fitsimage    =         ''   #   Name of input image FITS file
imagename    =         ''   #   Name of output CASA image
whichrep     =          0   #   Which coordinate representation (if multiple)
whichhdu     =          0   #   Which image (if multiple)
zeroblanks   =       True   #   If blanked fill with zeros (not NaNs)
overwrite    =      False   #   Overwrite pre-existing imagename
async        =      False   #   if True run in the background, prompt is freed
\end{verbatim}
\normalsize
For example, we can read the above image back in
\small
\begin{verbatim}
  importfits('ngc5921.usecase.image.fits','ngc5921.usecase.image.im')
\end{verbatim}
\normalsize

%%%%%%%%%%%%%%%%%%%%%%%%%%%%%%%%%%%%%%%%%%%%%%%%%%%%%%%%%%%%%%%%%
%%%%%%%%%%%%%%%%%%%%%%%%%%%%%%%%%%%%%%%%%%%%%%%%%%%%%%%%%%%%%%%%%
\section{Using the CASA Toolkit for Image Analysis}
\label{section:analysis.toolkit}

\begin{wrapfigure}{r}{2.5in}
  \begin{boxedminipage}{2.5in}
     \centerline{\bf Inside the Toolkit:}
     The image analysis tool ({\tt ia}) is the workhorse here.
     It appears in the User Reference Manual as the {\tt image}
     tool.  Other relevant tools for analysis and manipulation
     include {\tt measures} ({\tt me}), {\tt quanta} ({\tt qa})
     and {\tt coordsys} ({\tt cs}).
  \end{boxedminipage}
\end{wrapfigure}

Although this cookbook is aimed at general users employing the
tasks, we include here a more detailed description of doing
image analysis in the CASA toolkit.  This is because
there are currently only a few tasks geared towards image analysis,
as well as due to the breadth of possible manipulations that the
toolkit allows that more sophisticated users will appreciate.

To see a list of the {\tt ia} methods available, use the 
CASA {\tt help} command:
\small
\begin{verbatim}
CASA <1>: help ia 
--------> help(ia)
Help on image object:

class image(__builtin__.object)
 |  image object
 |  
 |  Methods defined here:
 |  
 |  __init__(...)
 |      x.__init__(...) initializes x; see x.__class__.__doc__ for signature
 |  
 |  __str__(...)
 |      x.__str__() <==> str(x)
 |  
 |  adddegaxes(...)
 |      Add degenerate axes of the specified type to the image`  : 
 |        outfile
 |        direction = false
 |        spectral  = false
 |        stokes
 |        linear    = false
 |        tabular   = false
 |        overwrite = false
 |      ----------------------------------------
 |  
 |  addnoise(...)

...

 |  
 |  unlock(...)
 |      Release any lock on the image`  : 
 |      ----------------------------------------
 |  
 |  ----------------------------------------------------------------------
 |  Data and other attributes defined here:
 |  
 |  __new__ = <built-in method __new__ of type object at 0x55d0f20>
 |      T.__new__(S, ...) -> a new object with type S, a subtype of T

\end{verbatim}
\normalsize
or for a compact listing use {\tt <TAB>} completion on {\tt ia.},
e.g.
\small
\begin{verbatim}
CASA <2>: ia.
Display all 101 possibilities? (y or n)
ia.__class__                ia.fitsky                   ia.newimagefromshape
ia.__delattr__              ia.fromarray                ia.open
ia.__doc__                  ia.fromascii                ia.outputvariant
ia.__getattribute__         ia.fromfits                 ia.pixelvalue
ia.__hash__                 ia.fromforeign              ia.putchunk
ia.__init__                 ia.fromimage                ia.putregion
ia.__new__                  ia.fromshape                ia.rebin
ia.__reduce__               ia.getchunk                 ia.regrid
ia.__reduce_ex__            ia.getregion                ia.remove
ia.__repr__                 ia.getslice                 ia.removefile
ia.__setattr__              ia.hanning                  ia.rename
ia.__str__                  ia.haslock                  ia.replacemaskedpixels
ia.adddegaxes               ia.histograms               ia.restoringbeam
ia.addnoise                 ia.history                  ia.rotate
ia.boundingbox              ia.imagecalc                ia.sepconvolve
ia.brightnessunit           ia.imageconcat              ia.set
ia.calc                     ia.insert                   ia.setboxregion
ia.calcmask                 ia.isopen                   ia.setbrightnessunit
ia.close                    ia.ispersistent             ia.setcoordsys
ia.continuumsub             ia.lock                     ia.sethistory
ia.convertflux              ia.makearray                ia.setmiscinfo
ia.convolve                 ia.makecomplex              ia.setrestoringbeam
ia.convolve2d               ia.maketestimage            ia.shape
ia.coordmeasures            ia.maskhandler              ia.statistics
ia.coordsys                 ia.maxfit                   ia.subimage
ia.decompose                ia.miscinfo                 ia.summary
ia.deconvolvecomponentlist  ia.modify                   ia.toASCII
ia.done                     ia.moments                  ia.tofits
ia.echo                     ia.name                     ia.topixel
ia.fft                      ia.newimage                 ia.toworld
ia.findsources              ia.newimagefromarray        ia.twopointcorrelation
ia.fitallprofiles           ia.newimagefromfile         ia.type
ia.fitpolynomial            ia.newimagefromfits         ia.unlock
ia.fitprofile               ia.newimagefromimage        
\end{verbatim}
\normalsize

A common use of the {\tt ia} tool is to do region statistics on
an image.  The {\tt imhead} task has {\tt mode='stats'} to do
this quickly over the entire image cube.  The tool can do this
on specific planes or sub-regions.  For example, in the Jupiter
6cm example script (\S~\ref{section:scripts.jupiter}), 
the {\tt ia} tool is used to get on-source and off-source statistics
for regression:
\small
\begin{verbatim}
# The variable clnimage points to the clean image name

# Pull the max and rms from the clean image
ia.open(clnimage)
on_statistics=ia.statistics()
thistest_immax=on_statistics['max'][0]
oldtest_immax = 1.07732224464
print ' Clean image ON-SRC max should be ',oldtest_immax
print ' Found : Max in image = ',thistest_immax
diff_immax = abs((oldtest_immax-thistest_immax)/oldtest_immax)
print ' Difference (fractional) = ',diff_immax

print ''
# Now do stats in the lower right corner of the image
box = ia.setboxregion([0.75,0.00],[1.00,0.25],frac=true)
off_statistics=ia.statistics(region=box)
thistest_imrms=off_statistics['rms'][0]
oldtest_imrms = 0.0010449
print ' Clean image OFF-SRC rms should be ',oldtest_imrms
print ' Found : rms in image = ',thistest_imrms
diff_imrms = abs((oldtest_imrms-thistest_imrms)/oldtest_imrms)
print ' Difference (fractional) = ',diff_imrms

print ''
print ' Final Clean image Dynamic Range = ',thistest_immax/thistest_imrms
print ''
print ' =============== '

ia.close()

\end{verbatim}
\normalsize

{\bf BETA ALERT:} Bad things can happen if you open some tools,
like {\tt ia}, in the Python command line on files and forget to
close them before running scripts that use the 
{\tt os.system('rm -rf <filename>)} call to clean up.  We are 
in the process of cleaning up cases like this where there can
be stale handles on files that have been manually deleted, but
for the meantime be warned that you might get exceptions
(usually of the ``{\tt SimpleOrderedMap-remove}'' flavor, or 
even Segmentation Faults and core-dumps!

%%%%%%%%%%%%%%%%%%%%%%%%%%%%%%%%%%%%%%%%%%%%%%%%%%%%%%%%%%%%%%%%%
%%%%%%%%%%%%%%%%%%%%%%%%%%%%%%%%%%%%%%%%%%%%%%%%%%%%%%%%%%%%%%%%%
